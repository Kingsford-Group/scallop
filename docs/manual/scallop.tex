\documentclass{llncs}
%
\usepackage{makeidx}  % allows for indexgeneration
\usepackage{pslatex}
\usepackage{graphicx}
\usepackage{amssymb}
%\usepackage{url}
\usepackage[obeyspaces]{url}

%\makeatletter
%\renewcommand{\section}{%
%  \@startsection{section}{1}{\z@}
%  {-14\p@ plus -5\p@}{4\p@}
%  {\normalfont\bfseries\fontsize{10}{11.5}\boldmath\raggedright\MakeUppercase}}
%\renewcommand{\subsection}{%
%  \@startsection{subsection}{2}{\z@}
%  {-8\p@ plus -3\p@}{2\p@}
%  {\raggedright\boldmath\bfseries\fontsize{9.5}{11.5}}}
%\renewcommand{\@listI}{\leftmargin\leftmargini \topsep\smallskipamount}
%\let\@listi\@listI
%\@listi
%\makeatother

% allow large figures on a page with little text
\renewcommand{\topfraction}{0.99}
\renewcommand{\bottomfraction}{0.99}
\renewcommand{\textfraction}{0.01}

\advance\textheight by 1cm

%
\begin{document}
%
\frontmatter          % for the preliminaries
%
\pagestyle{headings}  % switches on printing of running heads
%\addtocmark{Hamiltonian Mechanics} % additional mark in the TOC
%
%\tableofcontents
%
\mainmatter              % start of the contributions
%
\title{Scallop User Reference}
%
%\titlerunning{trajectory Graphs}  % abbreviated title (for running head) also used for the TOC unless \toctitle is used
%
\author{Mingfu Shao\inst{1} \and Carl Kingsford\inst{1}}
%
%\authorrunning{Mingfu Shao et al.} % abbreviated author list (for running head)
%
\institute{$^1${Computational Biology Department, Carnegie Mellon University}\\
\email{\{mingfu.shao, carlk\}@cs.cmu.edu}}

\maketitle              % typeset the title of the contribution

\section{Installation}
To install Scallop, you need to first download/compile a few software packages~(Samtools,
Boost, and GUROBI), setup the corresponding environmental variables, and then
compile the source code of Scallop.

\subsection{Install Samtools}
Download Samtools from \url{http://www.htslib.org/} with version 1.2 or higher.
Compile it to generate the htslib file \url{libhts.a}. Set environment variable
\url{HTSLIB} to indicate the directory of \url{libhts.a}. 
For example, for Unix platforms, add the following
statement to the file \url{~/.bash_profile}:\\
\makebox[0.9\textwidth][l]{\hspace{0.618cm}\url{export HTSLIB="/directory/to/your/htslib/htslib-1.2.1"} }

\subsection{Install Boost}
Download Boost from \url{http://www.boost.org}. Uncompress it
somewhere~(compiling and installing are not necessary). Set environment
variable \url{BOOST_HOME} to indicate the directory of Boost.
For example, for Unix platforms, add the following
statement to the file \url{~/.bash_profile}:\\
\makebox[0.9\textwidth][l]{\hspace{0.618cm}\url{export BOOST_HOME="/directory/to/your/boost/boost_1_60_0"} }

\subsection{Install GUROBI}
Download GUROBI from \url{http://www.gurobi.com/} and uncompress the
package somewhere~(compiling and installing are not required).
You need to apply an academic license to use
the full features of GUROBI~(Please refer to the GUROBI documentation for more information.)
After that, set two environment
variables, \url{GUROBI_HOME} and \url{GRB_LICENSE_FILE}, which indicates the directory of GUROBI, and
the location of your license file, respectively.
For example, for Unix platforms, add the following
two statements to the file \url{~/.bash_profile}:\\
\makebox[0.9\textwidth][l]{\hspace{0.618cm}\url{export GUROBI_HOME="/directory/to/your/gurobi/linux64"} }\\
\makebox[0.9\textwidth][l]{\hspace{0.618cm}\url{export GRB_LICENSE_FILE="/location/of/your/license/gurobi.lic"} }

\subsection{Compile Scallop}
Get the source code of Scallop through \url{git}:\\
\makebox[0.9\textwidth][l]{\hspace{0.618cm}\url{$git clone git@github.com:shaomingfu/scallop.git .}}\\
Execute the following commands to generate \url{Makefile} and compile:\\
\makebox[0.9\textwidth][l]{\hspace{0.618cm}\url{$cd src} }\\
\makebox[0.9\textwidth][l]{\hspace{0.618cm}\url{$aclocal} }\\
\makebox[0.9\textwidth][l]{\hspace{0.618cm}\url{$autoconf} }\\
\makebox[0.9\textwidth][l]{\hspace{0.618cm}\url{$autoheader} }\\
\makebox[0.9\textwidth][l]{\hspace{0.618cm}\url{$automake -a} }\\
\makebox[0.9\textwidth][l]{\hspace{0.618cm}\url{$./configure } }\\
\makebox[0.9\textwidth][l]{\hspace{0.618cm}\url{$make} }\\
The executable file \url{scallop} will be present at \url{src/src}.
You might want to link it into \url{bin} through\\
\makebox[0.9\textwidth][l]{\hspace{0.618cm}\url{$cd bin} }\\
\makebox[0.9\textwidth][l]{\hspace{0.618cm}\url{$ln -sf ../src/src/scallop .} }

\section{Command line}
The usage of Scallop is as follows:\\
\makebox[0.9\textwidth][l]{\hspace{0.618cm}
	\url{$./scallop -c config -i input.gtf  -a algo -o output.gtf}}\\

Parameter \url{config} configures the behavior of the algorithm.
There is such an example configure file at \url{bin/example.config}.

Currently we work on perfectly estimated splice graph, represented in a
\url{gtf} file with augmented expressions.  
One such example can be found at \url{bin/example.expression.gtf}.
With this file Scallop will first
build the splice graph, and then try to decompose the graph to
recover the transcripts as well as their corresponding abundances.

There are three options for \url{algo} parameter: \url{scallop1}, \url{scallop2}, and \url{greedy}.
With option of \url{scallop1}, the program will only run the core algorithm to partly
decompose the given splice graph, which will predict fewer transcripts but with
higher accuracy. With option of \url{scallop2}, the program will completely
decompose the given splice graph, using greedy algorithm following the core part of the algorithm.
With option of \url{greedy}, the program will only use greedy algorithm to fully decompose
the given splice graph.

The predicted transcripts will be written in parameter \url{output.gtf}.
%The bam-file is the reads/reference alignment file that can be generated by TopHat or Bowtie.
%{\bf Note:} we assume that the given bam-file has already been sorted.
%If not, you can do that using samtools:\\
%\makebox[0.9\textwidth][l]{\hspace{0.618cm}\url{$samtools sort your.bam -o prefix} }

%Currently two files will be generated by Scallop:
%the file with name \url{greedy.gtf} gives the predicted transcripts using the greedy algorithm;
%the file with name \url{iterat.gtf} gives the predicted transcripts using the iterated
%algorithm.

\section{Simulation and Evaluation}
We use Flux Simulator to simulate transcript expression. Before simulating, an annotation
file~(a \url{gtf} file) of a particular genome is required. Sometimes Flux Simulator crashes
with some \url{gtf} files for some format issue. To avoid this, you might want to fix the
format of the raw \url{gtf} file through using the script at \url{bin/fix.gtf.sh}:\\
\makebox[0.9\textwidth][l]{\hspace{0.618cm} \url{$./fix.gtf.sh raw.gtf > new.gtf}}\\

We also need to prepare a parameter file for Flux Simulator. There is such an example
parameter file at \url{bin/flux.exp.params}. Make sure that in this parameter file
REF\_FILE\_NAME is specified as the (fixed) \url{gtf} file we mentioned above.
Now we can run the Flux Simulator:\\
\makebox[0.9\textwidth][l]{\hspace{0.618cm} \url{$flux-simulator -p param-file -x}}\\

An expression file \url{profile}~(specified in the \url{param-file}) will be generated.
Now we need to merge the original \url{gtf} file with this expression file to create
the input \url{gtf} file for Scallop. We can do this by using \url{bin/merge.exp.pl}:\\
\makebox[0.9\textwidth][l]{\hspace{0.618cm} \url{$./merge.exp.pl new.gtf profile > input.gtf}}\\

To evaluate the preformance of predicted transcripts, we provided a tool
located at \url{gtfcompare/}~(you need to compile the source code):\\
\makebox[0.9\textwidth][l]{\hspace{0.618cm} \url{$./gtfcompare output.gtf input.gtf > summary}}\\

In the \url{summary} file, for each gene, it gives the number of transcripts in \url{output.gtf}
and \url{input.gtf} respectively, and the number of them that appears in both file.
If this common number is equal to the number for \url{output.gtf}, a \url{TRUE} is followed;
otherwise, a \url{FALSE} is followed. A summary line is given at the bottom line
of the \url{summary} file.

\end{document}
